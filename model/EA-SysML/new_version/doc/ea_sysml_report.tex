\documentclass{openetcs_article}
% Use the option "nocc" if the document is not licensed under Creative Commons
%\documentclass[nocc]{template/openetcs_article}
\usepackage{url}
\usepackage{booktabs}
\usepackage[hidelinks]{hyperref}
\usepackage{multicol}
\graphicspath{{./template/}{.}{./images/}{schema/pdf/}}
\begin{document}
\frontmatter
\project{openETCS}

%Please do not change anything above this line
%============================
% The document metadata is defined below

%assign a report number here
\reportnum{OETCS}

%define your workpackage here
\wp{Work-Package 7: ``Toolchain''}


\author{C\'ecile Braunstein}

\affiliation{Bremen University}

\title{Radio Communication Management \\
{\it SRS SUBSET-026-3.5} \\SysML Model}


% define the coverart
\coverart[width=350pt]{openETCS_EUPL.png}

\reporttype{Model Description}

%\begin{document}
\begin{abstract}
%define an abstract here
This document is a part of the model-evaluation project of WP7, it explores
SysML capabilities to provide a (semi)formal model of the SRS SUBSET-026.
\end{abstract}
\maketitle
\tableofcontents
\listoffiguresandtables
\newpage
% ----------------------------------------------------

\section{Short Introduction to Formalism and Tool}
\label{sec:short-intr-form}
This document describes the modeling  of the radio communication management. The
model has been made from the specification description of the SRS SUBSET-026-3.5
 baseline 3 as recommended by D2.5 Methods and tools benchmarking methodology
 \cite{D2-5}.


The formalism used  is SysML \cite{SysML}. More particularly we had used
block diagram, state charts and requirements diagram. 
SysML is a graphical language that extends UML for a customize version suitable
for system engineering. It may help modeling system within a board range of
system variety that may include hardware, software, data, personnel and
facilities. It supports the specification, analysis design, verification and
validation of complex system. 

Enterprise architect (EA) version 9.3 \cite{EA} has been used to implement the
model. Note that this tool is not open source but others tools  such as Papyrus
\cite{papyrus} provides SysML modeling capabilities and are evaluate by others
partners.
EA is a visual platform for designing and constructing
software systems, for business process modeling, and for more generalized
modeling purposes. it covers all aspects of the development cycle. The main
advantages is the requirement management and tracing, the team work and the
include versionning. The main cons : it is not an open source tool.

We had design a test model, this model aims at generate test cases and test
data. The test generator used here is the tool-suite RT-Tester Model-Based generator
(\textsc{Rtt-Mbt}) \cite{Peleska2011,rttmbtreport2011}. It provides sequences of
input data with timing constraints that are used for the stimulation of the
system under test (SUT), concurrently with generated test oracles. 
% ----------------------------------------------------

\section{Modeling Strategy}
\label{sec:modeling-strategy}
Starting from the specification SUBSET-026-chap3.5, we come up with the list of
requirements. The requirements are exactly the text of the SRS. 
This sub-part of the specification defines the management of the radio
communication, e.g. the protocol to follow in order to initiate, 
maintain  and terminate a radio communication. 
From the requirements list, we modeled the set of behaviors defined by direct translation 
into a state chart diagram. In parallel, we implemented  a requirement diagram that
links the requirement list and  the SRS to the model.

An example  figure \ref{fig:methodo-ex} shows an overview of the translation of the specification into a
SysML model. The rest of the document will give  more details on each
object presented. The basic idea is starting from the textual representation, we
can deduce: 
\begin{itemize}
\item Input events : "receives the system version" \\
\verb+MessageIn = SYS_VERSION+
\item Internal variables and constants : "The communication session shall be considered
as established"  \\
\verb+session == ESTABLISHED+
\item Outputs : "shall send a session established report" \\
\verb+MessageOut = SESSION_ESTABLISHED+.
\end{itemize}

\begin{figure}[htbp]
\centering
\includegraphics[width=\textwidth]{methodo_example}
\caption{\label{fig:methodo-ex} Methodology example}
\end{figure}

The radio communication management task is part of the set of on-board functions
(see SUBSET-026.4.5). The behavior of this function depends on the result of
others functions such as the function that determine of the ETCS Mode or Level.
This leads us to define an interface between this task
and the other tasks of the on board unit (see section \ref{sec:deta-model-descr}).

Moreover the model is designed for test generation purpose. The model of the
system under test (SUT)  has been
made together with the test execution environment (TE). Here the test
environment is really trivial since it is empty, it allows any possible value
for the inputs of the SUT. One may want to add some constrains on the input
signals or want to sketch some behavior such as an event never happened before
an other one. This may also be model as a state machine.
% ----------------------------------------------------

\section{Model Overview}
\label{sec:model-overview}
The model is composed of two packages : the \emph{system} and the
\emph{requirement}. The system package consists of the SUT model, the TE
model and a set of constants and enumeration types. The
requirement package contains all the requirement and a requirement diagram.

% ----------------------------------------------------

  \subsection{Block and state chart view}
  \input{interface}
% ----------------------------------------------------

  \subsection{Requirement view}
  \input{requirements}
% ----------------------------------------------------

\section{Model Benefits}
\label{sec:model-highlights}

In the previous sections, we had describes the use of SysML for modeling
SUBSET-026.3.5. Note that other aspect of the SRS as define in Deliverable D2.5
may also easily represent with SysML. The study here shows the 
modelisation of state charts and timeout. Moreover the arithmetic may be
represented as parametric diagram with constraints block. It is also possible to
use state charts to discretize the behavior of breaking curves.



The benefits of SysML for a semi formal modeling of the SUBSET-026-3.5.
\begin{itemize}
\item General-purpose  modeling languages
\item Can model different domain (arithmetic, state charts ...)
\item Easy export : As an OMG UML 2.0 profile, SysML models are designed to be
exchanged using the XML Meta-data Interchange (XMI) standard.
\item Open source licensee for SysML language
\item Requirements diagram to capture functional and/or performance requirements.
\item Requirements link the SRS to the model
\item Easy data structure definition
\item Customized language via profile, may be use for domain specific purpose
\end{itemize}
Enterprise Architect benefits~:
\begin{itemize}
\item Graphical modeling
\item Different view of the system in one tool
\item List view of all model elements
\item Author, date and status associated to each model element
\item Export/Import facilities
\item Different view for diagrams (Table views,  hierarchy or list view)
\end{itemize}

Some drawbacks~:
\begin{itemize}
\item SysML is only semi-formal
\item EA is not an  open-source software
\item EA needs to work with different tools or plug-in to animate and simulate
the models that are not always using the same XMI definition.
\item The semantic and a glossary should be defined before using SysML
\item EA No requirements table representation
\end{itemize}

% ----------------------------------------------------

\section{Detailed Model Description}
\label{sec:deta-model-descr}
% ----------------------------------------------------
  \subsection{Abstraction}
  \input{abstraction}

% ----------------------------------------------------
  \subsection{Inputs and outputs messages}
  \input{inputoutput}

% ----------------------------------------------------
  \subsection{Behaviour}
  \label{subsec:behavior}
  \input{behavior}

% ----------------------------------------------------
%  \subsection{Test cases generation}
%  \label{subsec:test_generation}
%  \input{rttester}


\bibliographystyle{plain}
\bibliography{model}
\end{document}


%%% Local Variables:
%%% mode: latex
%%% TeX-master: t
%%% End:
